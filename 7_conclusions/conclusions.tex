\chapter{Conclusions and Future work}
\label{chap:conc}
This thesis presents several contributions to the state of the art to address research objectives in the are of knowledge graph construction using declarative mapping languages. The contributions and identified future lines of work are summarized below.


\section{Achievements}
Constructing knowledge graphs from heterogeneous data sources is a complex data integration problem. Open research problems addressed in this thesis are: (i) the generation and interoperability of different mapping rules specifications to facilitate to users the KGC process, (ii) the creation of representative evaluation methods to provide an overview of the state of the art on the KGC engines and to understand their current limitations, (iii) as well as optimizations techniques to scale up the construction of virtual but also materialized KGs. 
 
 
The first objective of the this thesis is focused on define \textbf{representative features of a new knowledge graph construction generation systems}. This is done in Chapter \ref{chapter:mappig-translation}, where the \textit{mapping translation} concept is defined, adding a new layer into a KGC workflow. As we demonstrate with several use cases, exploiting the benefits of making interoperable different mapping languages specifications can enhance several steps of this process. The specific use case shown in this thesis is on the statistics domain, where we propose a set of new properties over the R2RML specification to improve the maintainability of the creation of the mapping rules in this domain. The ideas around this concept are also used over the different optimizations shown in Chapters \ref{chapter:virtual} and \ref{chapter:construction}.

The \textbf{exploitation of mapping rules to enhance the construction of virtual and materialized knowledge graphs} techniques is one of the main contributions of this thesis. To the best of our knowledge, the mapping driven optimizations techniques proposed in this work are the first ones that put the focus and exploit information from the semantic annotations. The heuristic based approaches proposed by Morph-CSV (Section \ref{chap6_morphgcsv}) and FunMap (Section \ref{chap7_funmap}) empirically demonstrate over several benchmark and use cases the importance of declarative annotations in a KGC process to efficiently deal with the heterogeneity of input data sources in the current web of data. Additionally, Morph-GraphQL (Section \ref{chap6_morphgraphql}) emphasizes the necessity of semantic web technologies, and more specific, the mapping rules, for avoiding data silos where non-semantic web approaches (e.g., GraphQL, API Rest, etc) are used to expose data on the web. Finally, SDM-RDFizer (Section \ref{chap7_rdfizer}) reveals the importance of well design physical data structures and their corresponding operators to scale-up the construction of knowledge graphs. Summarizing, we have identified the limitations of the proposals of the state of the art together with their open problems and we tackle them from a research perspective, highlighting that engineering solutions are not enough to solve complex data integration problem for constructing knowledge graphs.

To accomplish the second objective of this thesis, described as \textbf{representative evaluation systems for knowledge graph construction engines from heterogeneous data sources}, we present three different contributions. First, we analyze and extend the test cases presenting for RDB2RDF engines to coverage heterogeneous data sources, using RML as mapping language. In this manner, we can provide an overview of the compliance of the engines over this mapping language, which help user and practitioners to select an specific engine for their use cases. Second, we select and analyze the parameters that can impact in the performance and completeness of KGC engines. Our ambition is that the reported results of this contribution, inspire the community to define general testbeds that facilitate the understanding of the state of the art and the development of novel tools for constructing knowledge graphs at large scale. Following this ambition, we define the GTFS-Madrid-Bench, a benchmark for (virtual) KGC engines over the transport domain. Integrating the parameters defined in our previous work and defining a set of representative SPARQL queries, we propose the first benchmark that contributes to evaluate in a representative manner virtual KGC engines from one or multiple data sources and formats. We empirically test our approach over a set of heterogeneous KGC engines and identify multiple and promising future research work lines in this topic. Although the first and second contributions have been tested over materialized KGC engines and the third one over virtual KGC engines, notice that the contributions of this thesis are agnostic to the type of process to be performed, and can be used to test the capabilities of both approaches.


\section{Future Work}
In this section we describe those research problems that were not tackled during this thesis, due to time-permitting issues, or that were appearing as continuation of the proposed solutions in this work.

Aligned with the vision of the future generation of knowledge graph construction systems, we think that the mapping translation concept needs to be explored further, and this would allow a new range of KGC approaches that may be part of a new generation. In our opinion, the KGC community should see this variety of mapping languages not only as challenges (e.g., interoperability) but also, and mainly, as an opportunity for further research and development in this area, to address the need to cover more types of data sources while taking advantage of all the work that has been done in advanced aspects like query translation. Providing mapping translator services across mapping languages would bring further benefits and increase the availability of ontology-based data for its exploitation by search engines and query answering systems at Web scale. Additionally, the definition of a conceptual model describing the concepts of different mapping languages using the same vocabulary can be one of the first points to provide such translation services across different specifications. Finally, the analysis of the role of the users in the process of constructing knowledge graphs will be essential to develop robust and useful solutions in complex data integration environments.

One of the main future lines we have identify during this thesis, extending the contributions on the enhancement of KGC systems, is to define methodologies and techniques for an optimal physical design of knowledge graphs. The main idea is to be able to decide which parts of a KG have to be materialized or virtualized analyzing the features of the typical inputs of a KGC process (data, constraints, mapping rules, ontology). We believe that these methodologies will help to start to see the web as an integrated database that can be queried using Semantic Web technologies. Together with the application of the optimizations techniques proposed in this thesis over distributed environments, such as the ones proposed in \citep{endris2019ontario,mami2019squerall}, leverage the use of declarative KGC techniques to its next steps providing the basis for developing real-world knowledge graph applications.

For the evaluation systems, we need to extend the current proposals in order to coverage other be more flexibly to evaluate a KGC workflow, taking into account all the parameters that can have an impact in their behavior. Some examples of these possible future lines are: the inclusion of mapping rules with transformation functions, the adaption of mapping rules construction in a data integration system to isolated parameters from this input in the evaluation, or the improvement in creation of datasets at scale, exploiting the information from mapping rules or graph constraints (e.g, SHACL shapes). Finally, it is important to create evaluation systems that include a ground truth in order to test not only the performance and scalabitily of the engines but also other important features such as correctness and completeness.

Finally, the use of declarative and standard mapping rules and metadata descriptions makes possible the generalization of KGC engines and optimizations, avoiding ad-hoc and manual steps. It also incorporates a set of important benefits for these processes such as the improvement of its maintainability, readability, and understandability. We believe that this kind of solutions should be promoted in academic, industry and public organizations as good practices for data management on the web to for example, avoid to have data cemeteries such as the current open data portals. Our vision is that, analyzing the role of the users in complex data integration environments on the web, will help to understand how to promote and develop robust and useful semantic web solutions for constructing knowledge graphs at scale in distributing scenarios.

