\chapter{Methodology}

TBD
%Se consdera poco la integracion de subsimbolico a simbolico como razonamiento y no como modelo. 

% Siempre suele ser entre reglas-redes de neuronas. No considera las limitaciones existentes que se tienen que cumplir para que la integración sea correcta. 

%%RESTRICCIONES

%No consideramos modelos unificados donde ambos tienen la misma importancia, sino hibridos (revisar notación para que sea coherente), de manera que existe un modelo maestro y un modelo 'slave'. No se puede usar esta terminologia ya porque es un poco reicist. Importante diferenciar cuando se hace lo de la integracion de KBS a DL que se debe introducir INTERPRETABILIDAD (ser coherente con la notación dada en el SOTA). Todas las representaciones se van a hacer siguiendo el modelo de Harmelen y Ten Teije que ademas son mis colegas.


%%%CONSIDERAMOS LOS METODOS DESDE CUATRO PERSPECTIVAS: LIMITACIONES, CONSIDERACIONES, CONSTRAINTS, IMPACTO.