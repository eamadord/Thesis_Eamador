\chapter{Methodology}

TBD

%LIMITACIONES SOTA:
%Se consdera poco la integracion de subsimbolico a simbolico como razonamiento y no como modelo. 
%%
% Siempre suele ser entre reglas-redes de neuronas. No considera las limitaciones existentes que se tienen que cumplir para que la integración sea correcta. 

%%PLANTEAMIENTO DEL PROBLEMA

%HIPOTESIS:


%%OBJETIVOS
% Definir los parámetros necesarios para llevar a cabo la integración de KBS y DL


%%ASSUMPTIONS
% Los métodos propuestos sólo consideran la integración de KBS y DL entendiendo como parte de estas categorías los modelos del SOTA en esta parte
% Los parámetros descritos pueden ser modificados en el futuro si las limitaciones existentes en alguno de los modelos fueran resueltas sin la necesidad de integración


%%RESTRICCIONES

%No consideramos modelos unificados donde ambos tienen la misma importancia, sino hibridos (revisar notación para que sea coherente), de manera que existe un modelo maestro y un modelo 'slave'. No se puede usar esta terminologia ya porque es un poco reicist. Importante distinguir en integracion quien es el primary (EL QUE EJECUTA LA MAGRA) y el secondary (EL QUE MEJORA LA MAGRA). En el caso de la extraccion, el primario es EL QUE EXTRAE y el secondary es DEL QUE SE EXTRAE.
 
%Importante diferenciar cuando se hace lo de la integracion de KBS a DL que se debe introducir INTERPRETABILIDAD (ser coherente con la notación dada en el SOTA). Todas las representaciones se van a hacer siguiendo el modelo de Harmelen y Ten Teije que ademas son mis colegas.


%%%CONSIDERAMOS LOS METODOS DESDE CUATRO PERSPECTIVAS: LIMITACIONES, CONSIDERACIONES, CONSTRAINTS, IMPACTO.

%%%Los diagramas de las propuestas se van a hacer en notación van bekkum

%%Contribuciones: 
% 1. Método de introducción de KBS en DL
% 2. Método de inicialización basado en semántica para modelos de KGE
% 3. Metodo de introduccion de DL a KBS
% 4. Framework modular y adaptable a diferentes entornos de CBR mejorado con DL para la generacion de documentos.

%%RESEARCH METHODOLOGY
