\chapter{Introduction}
\label{chap:intro}
The Artificial Intelligence (AI) spectrum is vast and complex. The diversity amongst models, as well as its constant evolution, makes it difficult to precisely categorize the existing paradigms. From an input standpoint, two complimentary categories can be identified \citep{lieberman_symbolic_nodate}: symbolic and subsymbolic. The term \textit{symbolic} relates to those models whose input is human-readable and follow an inference process that can be understood by humans, while \textit{subsymbolic} models employ arithmetical representations for data representation and are opaque to the user. \cite{lieberman_symbolic_nodate} highlights the complementarity of both approaches. This general categorization is narrowed in \cite{hopgood_2009_knowledge-based}, where the AI spectrum is divided into knowledge-based systems (symbolic) and computationally intelligent systems (subsymbolic). With the breakthrough of deep learning \citep{raina_2009_gpu,glorot_2014_relu}, this classification is expanded to include these paradigms. 

Despite the variety of AI categorizations, the majority of them are founded in the idea of complementarity, which enforces the idea that the combination of antagonistic paradigms leads to new, enhanced approaches that combine the benefits of both elements. This idea is explored in neurosymbolic AI, which describes the set of methods designed for the integration of analogous AI models. Different methods for neurosymbolic system formulation have been proposed throughout the years, depicting the key features that should be exhibited \citep{mcgarry_hybrid_1999}, the criteria on which they should be rooted \citep{mira_neurosymbolic_2004}, or the principles that should be followed \citep{besold_neural-symbolic_2017}. Similar to AI paradigms, neurosymbolic approaches can be also categorized, mostly regarding the coupling degree between the integrated models \citep{medsker2020models,hilario_overview_nodate}.

From a general integration perspective, neurosymbolic approaches can be divided into two main categories: introduction (or insertion)\footnote{The terms ``introduction'' and ``insertion'' are used as synonyms throughout this thesis.}, and extraction. In insertion approaches, the different models are integrated under a unified framework, while in the extraction approaches one of the models is mined (or used to mine) from the other. Both neurosymbolic paradigms are present in a wide variety of tasks and scenarios. 

Regarding the introductory integration of deep learning and knowledge-based systems, the most prominent combination comprises rules and neural networks \citep{daniele_knowledge_2019,hatzilygeroudis_integrated_2010}. More recently, with the resurfacing of knowledge graphs, these paradigms are starting to gain relevance in this area. Knowledge graph completion \citep{nickel_review_ml_kg_2016,wang_kge_survey_2017} is one of the key challenges regarding knowledge graphs, where the goal is to find new potential elements from the information modelled in the graph. This task is usually performed by knowledge graph embedding models \citep{transe,distmult,crosse}, which are frequently based on deep learning paradigms. 

Most of the extraction approaches are encompassed in the area of explainable artificial intelligence, whose goal is to extract insights on the inference process of deep learning models. According to \cite{burkart_survey_2021}, AI models should be explainable and exhibit features such as \textit{trust, casualty, transferability,} or \textit{fair and ethical decision making}. These features are not intrinsic to deep learning models, which motivates extensive research on this direction.

The studied neurosymbolic design methods gave a grounding foundation on \textit{why} these methods are benefitial, and which features they should exhibit. However, no insights on \textit{how} these systems should be designs are provided. Therefore, there is a remarkable \textit{absence of guidelines} on the design of neurosymbolic models. Besides the existing motivations for integration, the precise limitations on the paradigms that motivate the need for a neurosymbolic model should be clearly outlined. Contextual and technical aspects should be also considered, such as the task to be performed, the available resources, or the required data types.

In addition to the theoretical aspects of neurosymbolic system design, there exists a lack of instantiation examples. As previously stated, most of the neurosymbolic systems solely rely on rule-based systems and neural networks. Subsequently, a considerable number of paradigms and potentially successful pairings between knowledge-based systems and deep learning approaches are dismissed. Showcasing the versatility, adaptability, and benefits of neurosymbolic integration across different open research challenges will further motivate its usage, while providing a set of references for their future reuse. 

Considering the previous aspects, the main goal of this thesis is to \textbf{propose a general design method for the integration of knowledge-based systems and deep learning models that comprises theoretical and technical aspects, and that can be instantiated on a wide variety of model pairings and case scenarios.} Therefore, the goal of this thesis is twofold and retaliates on both theoretical and practical aspects. From a theoretical standpoint, the limitations existing in previously existing neurosymbolic design methods are addressed. For this purpose, a general design method is formulated comprising four different dimensions: \textit{limitations, consideration, restrictions} and \textit{benefits}. Therefore, it integrates and extends the previously existing designs, giving a full design frame for the characterization of any possible instance. A specific set of parameters per dimension is devised for each of the potential interactions between knowledge-based systems and deep learning models. Several instances of the proposed design are presented to showcase its versatility and usability. These instances showcase a wide variety of model pairings across different scenarios. Moreover, these instances address existing open research challenges on their specific application domains, subsequently contributing to their resolution and materializing as research contributions of this thesis.
%%Contexto
%%Motivacion
%%Objetivo general (1p)

\section{Thesis Structure}
The content presented in this thesis is distributed into seven chapters. Chapters four, five, and six follow the same internal structure. The thesis is organized as follows:
\begin{itemize}
    \item \textit{\textbf{Chapter \ref{chap:soa}} (State-of-the-art)} presents the existing work that supports and motivates the research in this thesis. First, an overview of the artificial intelligence spectrum is provided, presenting the different existing categorizations and their comprising models. An overview on neurosymbolic integration is then presented, focusing on three main aspects: motivation, design methods, and categorization. Existing neurosymbolic approaches are then presented and categorized into the two types of interactions contemplated in this thesis: insertion and extraction.
    
    \item \textit{\textbf{Chapter \ref{chap:methodology}} (Problem Statement, Goals, and Research Methodology)} introduces the research problem attained in this thesis, as well as presenting an overview on the research hypotheses, objectives, assumptions, contributions, and restrictions formulated for its resolution. The followed research methodology is also presented.
    
    \item \textit{\textbf{Chapter \ref{chap:kbsintegrationdl}} (Knowledge-Based System Introduction into a Deep Learning Models)} explores the first of the covered integrations: the insertion of knowledge-based systems into deep learning models. The design method for this integration is described and instantiated by its application for knowledge graph completion.
    
    \item \textit{\textbf{Chapter \ref{chap:dlintegrationkbs}} (Deep Learning Introduction into Knowledge-Based Systems)} presents the second integration: deep learning introduction into knowledge-based systems. The design method for its integration is described and intanstiated for the problem of medical document generation.
    
    \item \textit{\textbf{Chapter \ref{chap:kbsextractiondl}} (Knowledge-Based System Extraction from Deep Learning Models)} outlines the third integration: knowledge-based system extraction from deep learning models. The design method for this integration is presented and instantiated on two different application scenarios: behavioral pattern extraction from black-box hyperpersonalization policies and insight and explanation extraction from knowledge graph embedding predictions.
    
    \item \textit{\textbf{Chapter \ref{chap:conc}} (Conclusions and Future Work)} draws the conclusions of the thesis, as well as future research lines. The research results generated during the thesis are also presented in this chapter.
\end{itemize}
