% Thesis Acknowledgements ------------------------------------------------


\begin{acknowledgementslong} %uncommenting this line, gives a different acknowledgements heading
%\begin{acknowledgements}      %this creates the heading for the acknowlegments
%Podría escribir que intentaré ser breve, pero quien esté leyendo esto seguramente me conoce lo suficiente para saber que no está en mi naturaleza. Y qué suerte que sea así, porque significa que tengo mucho que agradecer.
%%

%% El mayor de mis agradecimientos es sin duda para mis directores: Emilio y Daniel. Si ya fue un privilegio teneros como profesores, teneros como directores ha sido un verdadero regalo. 


%%HAY QUE AGRADECER  AL GRUPO DE FRANK, SOBRETODO A PETER
%%A SARA Y A LA GENTE DE ÁMSTERDAM
%Sara, gracias por tu generosidad, y por haber sido mi hogar lejos de casa. 

%%A LA GENTE DEL LABO QUE NO ESTÁ, LAS CHICAS DE LA PISCINA DE TERUEL

%%A LOS AMIGOS QUE HICISTE, A LOS QUE RECUPERASTE

%A IRENE

% Por supuesto, gracias a Tamara, por ser la hermana que la vida me regaló. Gracias por ser, por estar de forma incondicional, por tu infinita generosidad y por darme una segunda familia y unos amigos maravillosos. Una vida entera va a ser corta para agradecerte tanto.

%A mi abuela Geni, que aunque

%A mi madre, Geles. Siempre dices que tener un hijo significa llevarte cien disgustos por cada orgullo. Espero que este orgullo compense todos los disgustos que te haya dado (y los que seguro te seguiré dando). Gracias por enseñarnos a hacer del humor no una forma de vida, pero sí de supervivencia.    

%Y finalmente, como manda la tradición, estas últimas líneas son para el que es, y será siempre el hombre de mi vida, y de lo que haya después. A mi padre, Alberto, que además de su nariz y su pasión por la música, me enseñó que relativizar los problemas es el primer paso para resolverlos, y la importancia que tiene disfrutar de la vida. Qué largo se hace el camino sin ti, y cuánto me hubiera gustado seguir recorriéndolo de tu mano. Allí donde me estés esperando, ojalá estés orgulloso. 

%Me gustaría añadir los versos de la canción de Xoel López "Lodo", porque la he escuchado durante el proceso de escritura bastante y es como un buen recordatorio de que esto es el final del camino y de que es una etapa que abre las puertas de algo nuevo y bonito
% 
% \begin{flushright}
% \textit{"Y quizá hayas andado el camino ya \newline
% Cuando mires atrás \newline
% Si estás atrapado en las sombras \newline
% Aguarda, aguarda \newline
% Del lodo crecen las flores \newline
% Más altas, más altas." \newline
% -Xoel López}
% \end{flushright}
\end{acknowledgementslong}

% ------------------------------------------------------------------------


