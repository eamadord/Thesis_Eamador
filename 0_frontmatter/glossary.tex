% this file is called up by thesis.tex
% content in this file will be fed into the main document

% Glossary entries are defined with the command \item{1}{2}
% 1 = Entry name, e.g. abbreviation; 2 = Explanation
% You can place all explanations in this separate file or declare them in the middle of the text. Either way they will be collected in the glossary.

% required to print nomenclature name to page header
\chapter{Acronyms}
% \markboth{\MakeUppercase{\nomname}}{\MakeUppercase{\nomname}}
\begin{description}
\item[CBR]{Case-Based Reasoning}
\item[DL]{Deep Learning}
\item[GPU]{Graphical Processing Unit}
\item[KBS]{Knowledge-Based System(s)}
\item[KGC]{Knowledge Graph Completion}
\item[KGE]{Knowledge Graph Embedding(s)}
\item[NER]{Named Entity Recognition}
\item[NLP]{Natural Language Processing}
\item[TF-IDF]{Term Frequency-Inverse Document Frequency}

\end{description}
% ----------------------- contents from here ------------------------

% chemicals
% \item{DAPI}{4',6-diamidino-2-phenylindole; a fluorescent stain that binds strongly to DNA and serves to marks the nucleus in fluorescence microscopy} 
%\item{DEPC}{diethyl-pyro-carbonate; used to remove RNA-degrading enzymes (RNAases) from water and laboratory utensils}
%\item{DMSO}{dimethyl sulfoxide; organic solvent, readily passes through skin, cryoprotectant in cell culture}
%\item{EDTA}{Ethylene-diamine-tetraacetic acid; a chelating (two-pronged) molecule used to sequester most divalent (or trivalent) metal ions, such as calcium (Ca$^{2+}$) and magnesium (Mg$^{2+}$), copper (Cu$^{2+}$), or iron (Fe$^{2+}$ / Fe$^{3+}$)}



